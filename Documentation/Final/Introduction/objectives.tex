\section{Objectives}
\label{sec:objectives}

The objectives for this project are are follows:

\begin{itemize}
  \item \textbf{Align images using fuzzy entropy algorithms.} Images should be passed in, and it should be clear that an aligned version of the input images is calculated and output.
  \item \textbf{Answer relevant research questions associated with this project.} These are covered in Subsection \ref{ssec:research-qs}.
  \item \textbf{Create an application to streamline inputting images and viewing the output.} As this project uses light-weight, simpler fuzzy entropy algorithms to hopefully speed up processing time \textit{(see next objective)}, then the tool in which you run them should reflect this.
  \item \textbf{Create a quick application which can be used on anyone's laptop or PC.} Not many people outside of the research community use tools such as MATLAB, so to be able to run a simple executable program is important.
  \item \textbf{Research and implement a solution to remove medical markers from mammographic iamges.} As the \Gls{Congealing} algorithm looks to align the scans using grey-level pixel values, then the white medical markers in many mammograms create an issue as these will also try to align.
\end{itemize}

\section{Project Motivation}

Motivation for the project came from an interest into how computer science techniques, such as computer vision and image processing, could help the medical industry. When this project was initially outlined, the dataset of choice was left for the author to choose, paving the perfect pathway into combining these techniques with research into the prevention and early detection of breast cancer.

The project provided an opportunity for the author to explore new challenges which fell outside of their comfort zone, such as developing an application with both research and mathematical elements.
