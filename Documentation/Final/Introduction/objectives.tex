\section{Objectives}
\label{sec:objectives}

The Objectives for this project are are follows:

\begin{itemize}
  \item \textbf{Can images be aligned using Non-Probabilistic and Hybrid Entropy?} Through background research it would follow that there would be no issue in aligning images using fuzzy entropy techniques. However the implementation might be somewhat difficult.
  \item \textbf{Determine whether different fuzzy entropy alignment algorithms give different outputs.} And if so, could one be more useful than another? As the uncertainty in fuzzy entropy will help model different types of tissue, the way in which they assess uncertainty will affect the output image.
  \item \textbf{Create a tool to streamline inputting images and viewing the output.} As this project uses light-weight, simpler fuzzy entropy algorithms to hopefully speed up processing time \textit{(See next objective)}, then the tool in which you run them should reflect this.
  \item \textbf{Create a quick tool which can be used on anyone's laptop or PC.} Not many people outside of the research community use tools such as MATLAB, so to be able to run a simple executable program is important.
  \item \textbf{Research and implement a solution to remove medical markers from mammogram scans.} As the \Gls{Congealing} algorithm looks to align the scans using grey-level pixel values, then the white medical markers in many mammograms create an issue as these will also try to align.
  \item \textbf{Determine what advantages / disadvantages does each fuzzy entropy alignment metric entail?} One algorithm may be slower, but produce better results, so it is important to weigh up the speed versus the quality of the output.
\end{itemize}
