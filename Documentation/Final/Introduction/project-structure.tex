\section{Project Structure}

This section will give a brief overview of the structure of the project.

\subsection{Research}

The main piece of research to be undertaken in this project will be evaluating which Fuzzy Entropy algorithms will be light-weight and simple enough to be run quickly on a radiographer's own laptop. Typically, research implementations of Fuzzy Entropy algorithms tend to be complex, and therefore computationally expensive, something not ideal when a patient has a short time-slot with a radiographer.

\subsection{Software Implementation}

In order to assess the usefulness of basic fuzzy entropy algorithms in the alignment of mammographic scans, a tool must be built to handle the input images and all the output data. This tool will be created using MATLAB and it's Fuzzy Logic and Image Processing toolboxes.

The main functions of the tool will be:

\begin{itemize}
  \item Allow the user to input a large image containing all the scans they wish to align
  \item Allow the user to remove any medical markers as they see fit
  \item Allow the user to choose their alignment metric and number of iterations to run on the input images
  \item Output the final mean image, the adjusted input images (how they look after aligning) and the entropy of the final image set
\end{itemize}

\subsection{Testing}

The testing to be undertaken during this project will include scientific and software.

\subsubsection{Scientific testing}

This will be testing the output after the congealing process has been run using a fuzzy entropy alignment metric. One way to measure the result will be to evaluate the entropy value at the end of the alignment process - as the lower the entropy, the more aligned the images are. Another way in which to test the output of the experiments will be to visually inspect the final mean images produced to see how well aligned the input images are.

\subsubsection{Software testing}

Some software testing will be necessary to ensure the proper working of the tool developed for experimentation. Both Unit testing and acceptance testing off of the pre-defined user stories will be carried out.
