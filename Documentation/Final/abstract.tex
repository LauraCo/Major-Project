\thispagestyle{empty}

\begin{center}
    {\LARGE\bf Abstract}
\end{center}

More people are surviving breast cancer now than ever before. However research into prevention and early detection continues to drive towards the ultimate goal, that some day 99-100\% of sufferers will live to become survivors.

This project will assess whether leveraging image alignment techniques to align mammographic images using fuzzy entropy and shannon entropy metrics will produce an output which can be easily utilised by mammographers. The application handling the output must be lightweight, and the output should help the analysis of a patient's breast tissue density becomes quicker and easier. Classification of a patient's breast tissue is important as, should they fall within the higher-risk, denser tissue category, their care and screening plans going forward can be targeted to their needs. This paper will cover the hypotheses to be tested, the experiment tools \& methods leveraged and the application developed to aid mammographers.

The project saw encouraging results from the two fuzzy entropy algorithms implemented, with both aligning the data-sets with high accuracy, and with a low final entropy value. Additionally, low run-times for Shannon entropy and one of the fuzzy entropy alignment metrics provide promising signs of a usable, real-world application. Although the final results are somewhat subjective, this project has laid the groundwork for future development into fuzzy entropy image alignment methods.
