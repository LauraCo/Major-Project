\chapter{Background \& Objectives}

The Project is concerned with the alignment of multiple images using an image-alignment technique called Congealing \cite{joint-alignment}. The paper describes a method which utilises the Congealing algorithm to align both MNIST handwriting data and MRI Scans by reducing the pixel-wise uncertainty across a collection of images. This project aims to go further, aligning mammography scans and reducing the pixel-wise uncertainty using a number of different fuzzy entropy methods.

\section{Background}
In preparation for my Major Project extensive research was undertaken into the Fuzzy Entropy alignment metrics that would be chosen and implemented in the Congealing algorithm to align mammography scans. S. Al-sharhan's paper `Fuzzy Entropy: a Brief Survey' \cite{fuzzy-entropy} was instrumental in a brief comparison between Fuzzy Entropy metrics - allowing a simple way to compare and contrast mathematical differences and implementations.

Researchers have implemented many variations of the Congealing algorithm, with success in varying areas - however very little to no work has been done implementing the algorithm to assess mammography scans. Cox's `Least Squares Congealing' \cite{Cox_Sridharan_Lucey_Cohn_2008} was quickly disregarded given the project preference to select entropy-based alignment techniques, however did alert to the issue of performance for the project.

\section{Analysis}

\begin{itemize}
\item Does the use of Fuzzy Entropy alignment metrics improve the alignment of mammograms?
\item Do clinicians / radiographers / mammographers find the output at all useful?
\item What advantages / disadvantages does each fuzzy entropy alignment metric entail?
\end{itemize}
\section{Research Method}
You need to describe briefly the life cycle model or research method that you used. You do not need to write about all of the different process models that you are aware of. Focus on the process model or research method that you have used. It is possible that you needed to adapt an existing method to suit your project; clearly identify what you used and how you adapted it for your needs.
