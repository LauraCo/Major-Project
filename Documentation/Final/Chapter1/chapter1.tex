\chapter{Background \& Objectives}

The Project is concerned with the alignment of multiple images using an image-alignment technique called Congealing \cite{joint-alignment}. The paper describes a method which utilises the Congealing algorithm to align both MNIST handwriting data and MRI Scans by reducing the pixel-wise uncertainty across a collection of images. This project aims to go further, aligning mammography scans and reducing the pixel-wise uncertainty using a number of different fuzzy entropy methods.

\section{Background}
In preparation for my Major Project extensive research was undertaken into the Fuzzy Entropy alignment metrics that would be chosen and implemented in the Congealing algorithm to align mammography scans. S. Al-sharhan's paper `Fuzzy Entropy: a Brief Survey' \cite{fuzzy-entropy} was instrumental in a brief comparison between Fuzzy Entropy metrics - allowing a simple way to compare and contrast mathematical differences and implementations.

Researchers have implemented many variations of the Congealing algorithm, with success in varying areas - however very little to no work has been done implementing the algorithm to assess mammography scans. Cox's `Least Squares Congealing' \cite{Cox_Sridharan_Lucey_Cohn_2008} was quickly disregarded given the project preference to select entropy-based alignment techniques, however did alert to the issue of performance for the project.

The motivation behind choosing mammograms as the input data of choice was an interest into how computer systems and machine learning can help in the medical sector.

\section{Analysis}

\subsection{Task composition}

\subsubsection{Pixel Membership}

From the analysis of the planned Fuzzy Entropy algorithms, one major task to be undertaken would be to calculate the membership of each pixel. Membership stems from Fuzzy set theory \textbf{Check this?} which acknowledges that an element can be part of more than one group to a certain degree, not just one or the other. One common example of this is listing someone as `Short', `Average' or `Tall' in height. If a tall person is someone over 6 feet in height, would a person who measured 5foot 11inches not be classified as tall? In fuzzy set theory, they would be be a certain degree of tall, and a certain degree of average, with the highest membership likely to win out when categorising their height.

There are two common methods to modeling degrees of membership. The first is to manually define the categorty boundaries, so in the case of trapezium functions, the two bases and the two shoulders as in Figure x \textbf{include a diagram here showing?}. The other solution would be to iterate over the values you have and to computationally build the an even distribution throughout your membership functions. Whilst this is the preferred method for being dynamic in it's calculations, it is also more computationally expensive as pre-processing of the image would have to be completed before the Congealing algorithm could be run.

Taking this into account, for grey-level pixel values, ranging from 0 (black) to 255 (white), three trapezium functions would be sufficient, therefore modeling `Low', `Medium' and `High' grey-level values. The bases and shoulders would be statically defined by the User, however would be open for editing between running the Congealing code for experimentation purposes. For Non-Probabilistic entropy the highest membership for each pixel from each of the three trapeziums would be taken as the membership degree.

\subsection{Research questions}
\begin{itemize}
\item Does the use of Fuzzy Entropy alignment metrics improve the alignment of mammograms?
\item Do clinicians / radiographers / mammographers find the output at all useful?
\item What advantages / disadvantages does each fuzzy entropy alignment metric entail?
\end{itemize}

\section{Research Method}
%You need to describe briefly the life cycle model or research method that you used. You do not need to write about all of the different process models that you are aware of. Focus on the process model or research method that you have used. It is possible that you needed to adapt an existing method to suit your project; clearly identify what you used and how you adapted it for your needs.

\textbf{Literature review}
