%TC:ignore
\chapter{Third-Party Code and Libraries}

%If you have made use of any third party code or software libraries, i.e. any code that you have not designed and written yourself, then you must include this appendix.

%As has been said in lectures, it is acceptable and likely that you will make use of third-party code and software libraries. The key requirement is that we understand what is your original work and what work is based on that of other people.

%Therefore, you need to clearly state what you have used and where the original material can be found. Also, if you have made any changes to the original versions, you must explain what you have changed.

%As an example, you might include a definition such as:

%Apache POI libraryThe project has been used to read and write Microsoft Excel files (XLS) as part of the interaction with the clients existing system for processing data. Version 3.10-FINAL was used. The library is open source and it is available from the Apache Software Foundation. The library is released using the Apache License . This library was used without modification.

%\section{example section}

\section{Congealing Code}

The project focused on extending the existing Congealing Code implemented by Learned Miller et al in 2005. A Congealing demo is available on the Congealing website \cite{Learned-Miller} which is open for experimentation. The original demo code was modified and extended to be able to read in mammograms and to work with 2 Fuzzy Entropy algorithms.

\section{MATLAB}

MATLAB was the main implementation tool for this project \cite{MATLAB:2016}. It allowed the developer to leverage several built-in functions, such as \texttt{max} and \texttt{bsxfun} - the former to find the maximum value between two number, and the latter to apply a function (such as max) to all elements in an array. This streamlined the development process by ensuring that menial tasks could often be reduced down to a one-line implementation, especially when leveraging the extra Toolboxes provided by MATLAB. This is also covered in the main body of the report in Subsection \ref{ssec:matlab} about implementation tools.

\subsection{Image Processing Toolbox}

The Image Processing Toolbox \cite{image_toolbox} allowed the developer to leverage ready-made algorithms and functions vital for pre-processing and processing of images. Functions utilised in this project include ensuring the image was converted to grey-scale, reading and writing image data from/to files and obtaining image information, such as size and image type.

\subsection{Fuzzy Logic Toolbox}

The Fuzzy Logic Toolbox \cite{fuzzy_toolbox} was primarily used for modeling the fuzzy set membership functions utilised by Non-Probabilistic and Hybrid entropy - using functions such as \texttt{trapmf} and \texttt{evalmf}. More information is covered in the main body of the report in Subsection \ref{ssec:member-imp}.

%TC:endignore
