\documentclass[11pt,fleqn,twoside]{article}
\usepackage{makeidx}
\makeindex
\usepackage{palatino} %or {times} etc
\usepackage{plain} %bibliography style 
\usepackage{amsmath} %math fonts - just in case
\usepackage{amsfonts} %math fonts
\usepackage{amssymb} %math fonts
\usepackage{lastpage} %for footer page numbers
\usepackage{fancyhdr} %header and footer package
\usepackage{mmpv2} 
\usepackage{url}

% the following packages are used for citations - You only need to include one. 
%
% Use the cite package if you are using the numeric style (e.g. IEEEannot). 
% Use the natbib package if you are using the author-date style (e.g. authordate2annot). 
% Only use one of these and comment out the other one. 
\usepackage{cite}
%\usepackage{natbib}

\begin{document}

\name{Laura Collins}
\userid{lac32}
\projecttitle{Entropy-based metrics for joint image alignment}
\projecttitlememoir{Your project title (shorter form, if necessary)} %same as the project title or abridged version for page header
\reporttitle{Outline Project Specification}
\version{0.1}
\docstatus{Draft}
\modulecode{CS39440}
\degreeschemecode{GH7P}
\degreeschemename{Artificial Intelligence And Robotics}
\supervisor{Neil MacParthalain} % e.g. Neil Taylor
\supervisorid{ncm}
\wordcount{}

%optional - comment out next line to use current date for the document
%\documentdate{10th February 2014} 
\mmp

\setcounter{tocdepth}{3} %set required number of level in table of contents


%==============================================================================
\section{Project description}
%==============================================================================

The Project builds upon the Congealing code developed by Erik Learned-Miller from the University of Massachusetts \cite{joint-alignment}. In the paper Learned-Miller utilises the Congealing algorithm to align both MNIST handwriting data and MRI Scans by reducing the variability between each pixel within a stack of images. For example, the algorithm will analyse the properties of the upper-leftmost pixel in image 1, then in image 2 and through each image in the set. This variation between pixels is known as Entropy, and thus the objective of Congealing is to minimise the Entropy across a set of images.

The aim of this project will be to implement the Congealing algorithm using not only Entropy, as in Learned-Miller's application, but using other metrics such as Fuzzy Entropy, and Least Squares as implemented in \cite{least-squares-congealing}. As for the data to feed into the application, my image set of choice will be Mammography scans, however due to their large file size initial experimentation will be run using the MNIST handwriting data set.

Aligning mammography scans would help build an average of the image set. With this is in mind it is possible to build an average for the 4 breast tissue composition categories outlined by BI-RADS (Breast Imaging-Reporting and Data System) - with 1 being fatty tissue (lowest risk) and 4 being dense tissue (highest risk) \cite{bi-rads}. Approaching this alignment of images using different metrics would output different a range of output results, and further investigation could be undertaken to ascertain the best alignment metric.

Finally, the resulting software tool could be used by Radiologists in assessing the tissue variation between the 4 categories and could also be used to build a classifier model.

%==============================================================================
\section{Proposed tasks}
%==============================================================================

\subsection{Research into previous work}
Work has been done using the Congealing method on 3D MRI scans on brightness alignment. 

\subsection{Integrating existing Congealing algorithm}

The Congealing algorithm has been implemented in both C++ and Objective-C. As my programming language of choice is Java, I will need to take one of the implementations and wrap it to be able to integrate it with the rest of my code. 

I had an early decision to make as to whether I would program the entire project in C++ to accommodate the Congealing code, whether I would write my own implementation of Congealing in Java (this would ultimately be a great extra project, but I worry it would take away from my final goal) or finally whether I would wrap the existing open-source code by MIT and integrate it.

\subsection{Graphical User Interface}

%==============================================================================
\section{Project deliverables}
%==============================================================================



\nocite{*} % include everything from the bibliography, irrespective of whether it has been referenced.

% the following line is included so that the bibliography is also shown in the table of contents. There is the possibility that this is added to the previous page for the bibliography. To address this, a newline is added so that it appears on the first page for the bibliography. 
\newpage
\addcontentsline{toc}{section}{Initial Annotated Bibliography} 

%
% example of including an annotated bibliography. The current style is an author date one. If you want to change, comment out the line and uncomment the subsequent line. You should also modify the packages included at the top (see the notes earlier in the file) and then trash your aux files and re-run. 
%\bibliographystyle{authordate2annot}
\bibliographystyle{IEEEannot}
\renewcommand{\refname}{Annotated Bibliography}  % if you put text into the final {} on this line, you will get an extra title, e.g. References. This isn't necessary for the outline project specification. 
\bibliography{mmp} % References file

\end{document}
